\documentclass{beamer}
\usetheme{CambridgeUS}

\usepackage[utf8]{inputenc}
\usepackage[english]{babel}
\usepackage{amsthm}
\usepackage{tikzsymbols}
\usepackage{parskip}

%These for for continuing numbering between frames
\newcounter{saveenumi}
\newcommand{\seti}{\setcounter{saveenumi}{\value{enumi}}}
\newcommand{\conti}{\setcounter{enumi}{\value{saveenumi}}}

%\newtheorem{theorem}{Theorem}[section]
%\newtheorem{corollary}{Corollary}[theorem]
%\newtheorem{lemma}[theorem]{Lemma}
%\theoremstyle{definition}
%\newtheorem{definition}{Definition}[section]
\newtheorem{prop}[theorem]{Proposition}

%Information to be included in the title page:
\title{Section 3.4}
\subtitle{Introduction to Proofs}
\author{Dr. Ryan Johnson}
\institute{Grace College}
\date{Spring 2021}


%%%%%%%%%%%%%%%%%%%%%%%%%%%%%%%%%%%%%%%%%%%%%%%%%%%%%%%%%%%%%%%%%%%
\begin{document}
\begin{frame}[plain]
    \maketitle
\end{frame}

\begin{frame}{The Jesus Creed}
\Large{
"Hear, O Israel, the Lord our God, the Lord is one.\\
You shall love the Lord your God\\
\;\; with all your heart, with all your soul,\\
\;\; with all your mind, and with all your strength.\\
The second is this: Love your neighbor as yourself.\\
There is no commandment greater than these."
}
\end{frame}

\begin{frame}{Preview Activity 1}
    \begin{enumerate}
        \item Complete a truth table to show that
        \[
        (P \vee Q) \to R \equiv (P \to R) \wedge (Q \to R)
        \] \pause
        \item Suppose that you are trying to prove a statement that is written in the form $(P \vee Q) \to R$. Explain why you can complete this proof by writing separate and independent proofs of $P \to Q$ and $Q \to R$. \pause
        \item Now consider the following proposition:
        \begin{prop}
            For all integers x and y, if xy is odd, then x is odd and y is odd.
        \end{prop}
        Write the contrapositive of this proposition.
        \seti
    \end{enumerate}
\end{frame}

\begin{frame}{Preview Activity 1}
    \begin{enumerate}
        \conti
        \item Now prove that if x is an even integer, then xy is an even integer. Also, prove that if y is an even integer, then xy is an even integer. \pause
        \item Use the results proved in part (4) and the explanation in part (2) to explain why we have proved the contrapositive of the proposition in part (3).
    \end{enumerate}
\end{frame}

\begin{frame}{Preview Activity 2}
    \begin{prop}
        If n is an integer, then $n^2 + n$ is an even integer.
    \end{prop} \pause
    \begin{enumerate}
        \item Complete the proof for the following proposition:
        \begin{prop}
            If n is an even integer, then $n^2 + n$ is an even integer.
        \end{prop} \pause
        \item Complete the proof for the following proposition:
        \begin{prop}
            If n is an odd integer, then $n^2 + n$ is an even integer.
        \end{prop} \pause
        \item Explain why the proofs of the two propositions above can be used to construct a proof of the original Proposition.
    \end{enumerate}
\end{frame}

\begin{frame}{Progress Check 3.21}
    Notes: common times we need to break things into cases, and writing guidelines for cases on p133-134. \pause
    
    Complete the proof of the following proposition:
    \begin{prop}
        For each integer $n$, $n^2 - 5n + 7$ is an odd integer.
    \end{prop} 
\end{frame}

\begin{frame}{Progress Check 3.24*}
    \begin{definition}
    For $x \in \mathbb{R}$, we define $|x|$, called the \textbf{absolute value of} x by
    \[
    |x| = \begin{cases}
    x, & \text{if } x \geq 0;\\
    -x, & \text{if } x < 0.
    \end{cases}
    \]
    \end{definition}
\pause

    Solve the following equations
    \begin{enumerate}
        \item $|t| = 12$.
        \item $|t+3| = 5$.
        \item $|t-4| = \frac{1}{5}$
        \item $|3t - 4| = 8$.
    \end{enumerate}
\end{frame}

\begin{frame}{Class Activity}
    \begin{enumerate}
        \item Prove Part (2) of Proposition 3.23 from the book
        \[ \text{For each } x \in \mathbb{R}, |-x| = |x|. \]
        \item Prove Part (2) of Theorem 3.25
        \[ \text{For all real numbers } x \text{ and } y, |xy| = |x| |y| \]
    \end{enumerate}
\end{frame}

\end{document}


