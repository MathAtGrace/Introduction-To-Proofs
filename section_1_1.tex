\documentclass{beamer}
\usetheme{CambridgeUS}

\usepackage[utf8]{inputenc}
\usepackage[english]{babel}
\usepackage{amsthm}

%\newtheorem{theorem}{Theorem}[section]
%\newtheorem{corollary}{Corollary}[theorem]
%\newtheorem{lemma}[theorem]{Lemma}
%\theoremstyle{definition}
%\newtheorem{definition}{Definition}[section]

%Information to be included in the title page:
\title{Section 1.1}
\subtitle{Introduction to Proofs}
\author{Dr. Ryan Johnson}
\institute{Grace College}
\date{Spring 2021}

%%%%%%%%%%%%%%%%%%%%%%%%%%%%%%%%%%%%%%%%%%%%%%%%%%%%%%%%%%%%%%%%%%%
\begin{document}
\begin{frame}[plain]
    \maketitle
\end{frame}

\begin{frame}{The Jesus Creed}
\Large{
"Hear, O Israel, the Lord our God, the Lord is one.\\
You shall love the Lord your God\\
\;\; with all your heart, with all your soul,\\
\;\; with all your mind, and with all your strength.\\
The second is this: Love your neighbor as yourself.\\
There is no commandment greater than these."
}
\end{frame}

\begin{frame}{Statements}
	\begin{definition}
		In mathematics, a \textbf{statement} is a declarative sentence that is either true or false but not both. A statement is sometimes called a \textbf{proposition}.
	\end{definition}
\end{frame}

\begin{frame}{Progress Check 1.1 (Statements)}
	Which of the following sentences are statements? Do not worry about determining
	whether a statement is true or false; just determine whether each sentence is a
	statement or not.
	\begin{enumerate}
		\item \(3 + 4 = 8\)
		\item \(2 \cdot 7 + 8 = 22\)
		\item \( (x-1) = \sqrt{x + 11}\)
		\item \(2x + y = 7\) \pause
		\item There are integers $x$ and $y$ such that $2x + 5y = 7$.
		\item There are integers $x$ and $y$ such that $23x + 37y = 52$.
		\item Given a line $L$ and a point $P$ not on that line, there is a unique line through
		$P$ that does not intersect $L$.
		\item \( (a+b)^3 = a^3 + 3a^2b + 3ab^2 + b^3\).
		\item \( (a+b)^3 = a^3 + 3a^2b + 3ab^2 + b^3\) for all real numbers $a$ and $b$.
	\end{enumerate}
\end{frame}

\begin{frame}{Progress Check 1.1 (Continued)}
	\begin{enumerate}
		\item[10.] The derivative of $f(x) = \sin x$ is $f'(x) = \cos x$.
		\item[11.] Does the equation $3x^2 - 5x - 7 = 0$ have two real number solutions?
		\item[12.] If $ABC$ is a right triangle with right angle at vertex $B$, and if $D$ is the
		midpoint of the hypotenuse, then the line segment connecting vertex $B$ to $D$
		is half the length of the hypotenuse.
		\item[13.] There do not exist three integers $x$, $y$, and $z$ such that $x^3 + y^3 = z^3$.
	\end{enumerate}
\end{frame}

\begin{frame}{How Do We Decide If a Statement is True or False?}
	Techniques of Exploration
	\begin{itemize}
		\item Guesswork and conjectures
		\item Examples (Picking appropriate examples is important)
		\item An example that shows a statement is false is called a \textbf{counterexample}.
		\item Prior knowledge (e.g. the conjecture $\sin(2x) = 2\sin(x)$).
		\item Cooperation and brainstorming.
	\end{itemize}
\end{frame}

\begin{frame}{Progress Check 1.2 (Explorations)}
	Use the techniques of exploration to investigate each of the following statements.
	Can you make a conjecture as to whether the statement is true or false? Can you
	determine whether it is true or false?
	\begin{enumerate}
		\item \((a+b)^2 = a^2 + b^2\) for all real numbers $a$ and $b$.
		\item There are integers $x$ and $y$ such that $2x + 5y = 41$.
		\item If $x$ is an even number, then $x^2$ is an even number.
		\item If $x$ and $y$ are odd integers, then $x \cdot y$ is an odd integer.
	\end{enumerate}
\end{frame}

\begin{frame}{Conditional Statements}
	\begin{definition}
		A \textbf{conditional statement} is a statement that can be written in
		the form ``If $P$ then $Q$,'' where $P$ and $Q$ are sentences. For this conditional
		statement, $P$ is called the \textbf{hypothesis} and $Q$ is called the \textbf{conclusion}.
	\end{definition}

	We will sometimes use the notation $P \to Q$ to represent ``If P then Q.''
\end{frame}

\begin{frame}{Truth Tables}
	The box below is called a \textbf{truth table.}  Fill in the missing values.
	
	\begin{center}
		\begin{tabular}{|c|c|c|}
			\hline
			$P$ & $Q$ & $P \to Q$\\ \hline
			T & T & \\ \hline
			T & F & \\ \hline
			F & T & \\ \hline
			F & F & \\ \hline
		\end{tabular}
	\end{center}	
\end{frame}

\begin{frame}{Progress Check 1.4 (Explorations with Conditional Statements)}
	\begin{enumerate}
		\item Consider the following sentence:
		\begin{center}
			If $x$ is a positive real number,\\ then $x^2 + 8x$ is a positive real number.
		\end{center}
	(This is not a statement, but we can infer from the context that there is an implied ``for all real numbers $x$'')
	\begin{enumerate}
		\item Notice that if $x = -3$, then $x^2 + 8x = -15$, which is negative. Does
		this mean that the given conditional statement is false? \pause
		\item Notice that if $x = 4$, then $x^2 + 8x = 48$, which is positive. Does this
		mean that the given conditional statement is true? \pause
		\item Do you think this conditional statement is true or false? Record the
		results for at least five different examples where the hypothesis of this
		conditional statement is true.
	\end{enumerate}
	\end{enumerate}
\end{frame}

\begin{frame}{Progress Check 1.4 (Continued)}
	\begin{enumerate}
		\item[2.] ``If $n$ is a positive integer, then $(n^2 - n + 41)$ is a prime number.'' (Remember
		that a prime number is a positive integer greater than 1 whose only positive
		factors are 1 and itself.)
		
		To explore whether or not this statement is true, try using (and recording
		your results) for $n=1$, $n=2$, $n = 3$, $n=4$, $n=5$, and $n=10$. Then
		record the results for at least four other values of $n$. Does this conditional
		statement appear to be true?
	\end{enumerate}
\end{frame}

\begin{frame}{Progress Check 1.5 (Working with a Conditional Statement)}
	The following statement is a true statement, which is proven in many calculus texts.
	\begin{center}
		If the function $f$ is differentiable at $a$,\\ then the function $f$ is continuous at $a$.
	\end{center}
	Using only this true statement, is it possible to make a conclusion about the function in each of the following cases?
	\begin{enumerate}
		\item It is known that the function $f$ , where $f(x) = \sin x$, is differentiable at 0. \pause
		\item It is known that the function $f$, where $f(x) = \sqrt[3]{x}$ is not differentiable at 0.
	\end{enumerate}
\end{frame}

\begin{frame}{Progress Check 1.5 (Continued)}
	The following statement is a true statement, which is proven in many calculus texts.
	\begin{center}
		If the function $f$ is differentiable at $a$,\\ then the function $f$ is continuous at $a$.
	\end{center}
	Using only this true statement, is it possible to make a conclusion about the function in each of the following cases?
	\begin{enumerate}
		\item[3.] It is known that the function $f$, where $f(x) = |x|$, is continuous at 0. \pause
		\item[4.] It is known that the function $f$, where $f(x) = \dfrac{|x|}{x}$ is not continuous at 0.
	\end{enumerate}
\end{frame}

\begin{frame}{The Numbers We Use}
	\begin{enumerate}
		\item The numbers we will use the most are the \textbf{real numbers}, and we will use the symbol $\mathbb{R}$ for these numbers.
		\item The \textbf{rational numbers} are real numbers that can be written as fractions of integers.  We will use the symbol $\mathbb{Q}$ for the rationals.
		\item All real numbers that are not rational are called \textbf{irrational numbers}.
		\item The \textbf{natural numbers} $\mathbb{N}$ are all of the counting numbers 1,2,3,...
		\item The \textbf{integers} $\mathbb{Z}$ are the positive and negative naturals and zero.
	\end{enumerate}
\end{frame}

\begin{frame}{Closure Properties}
	\begin{itemize}
		\item The integers $\mathbb{Z}$ are \textbf{closed} under addition, subtraction, and multiplication.  This means
		\begin{itemize}
			\item If $x$ and $y$ are integers, then $x + y$ is an integer;
			\item If $x$ and $y$ are integers, then $x - y$ is an integer;
			\item If $x$ and $y$ are integers, then $x \cdot y$ is an integer;
		\end{itemize}
	\end{itemize}
\end{frame}

\begin{frame}{Progress Check 1.7 (Closure Properties)}
	\begin{enumerate}
		\item Is the set of rational numbers closed under addition? Explain.
		\item Is the set of integers closed under division? Explain.
		\item Is the set of rational numbers closed under subtraction? Explain.
	\end{enumerate}
\end{frame}

\end{document}
