\documentclass{beamer}
\usetheme{CambridgeUS}

\usepackage[utf8]{inputenc}
\usepackage[english]{babel}
\usepackage{amsthm}
\usepackage{tikzsymbols}
\usepackage{parskip}

%\newtheorem{theorem}{Theorem}[section]
%\newtheorem{corollary}{Corollary}[theorem]
%\newtheorem{lemma}[theorem]{Lemma}
%\theoremstyle{definition}
%\newtheorem{definition}{Definition}[section]
\newtheorem{prop}[theorem]{Proposition}

%Information to be included in the title page:
\title{Section 2.4}
\subtitle{Introduction to Proofs}
\author{Dr. Ryan Johnson}
\institute{Grace College}
\date{Spring 2021}

%%%%%%%%%%%%%%%%%%%%%%%%%%%%%%%%%%%%%%%%%%%%%%%%%%%%%%%%%%%%%%%%%%%
\begin{document}
\begin{frame}[plain]
    \maketitle
\end{frame}

\begin{frame}{The Jesus Creed}
\Large{
"Hear, O Israel, the Lord our God, the Lord is one.\\
You shall love the Lord your God\\
\;\; with all your heart, with all your soul,\\
\;\; with all your mind, and with all your strength.\\
The second is this: Love your neighbor as yourself.\\
There is no commandment greater than these."
}
\end{frame}

\begin{frame}{Preview Activity 1}
    \begin{enumerate}
        \item $(\forall a \in \mathbb{R}) (a + 0 = a)$. \pause \hspace{.5 in} True \pause
        \item $3x - 5 = 9$ \pause \hspace{.5 in} $\{\frac{14}{3}\}$  \pause
        \item $\sqrt{x} \in \mathbb{R}$. \pause \hspace{.5 in} $\{x \in \mathbb{R} \; | \; x \geq 0\}$   \pause
        \item $\sin(2x) = 2\sin(x)\cos(x)$ \pause \hspace{.5 in} $\mathbb{R}$  \pause
        \item $(\forall x \in \mathbb{R})(\sin(2x) = 2\sin(x)\cos(x))$ \pause \hspace{.5 in} True \pause
        \item $(\exists x \in \mathbb{R})(x^2 + 1 = 0)$ \pause \hspace{.5 in} False \pause
        \item $(\forall x \in \mathbb{R})(x^3 \geq x^2)$ \pause \hspace{.5 in} False \pause
        \item $x^2 + 1 = 0$. \pause \hspace{.5 in} $\{ \} = \emptyset$ \pause
        \item If $x^2 \geq 1$ then $x \geq 1$ \pause \hspace{.5 in} $\{x \in \mathbb{R} \; | \; x \geq -1\}$ \pause
        \item $(\forall x \in \mathbb{R})$(If $x^2 \geq 1$ then $x \geq 1$) \pause \hspace{.5 in} False
    \end{enumerate}
\end{frame}

\begin{frame}{Preview Activity 2: Quantifiers}
	\begin{enumerate}
		\item Consider the following statement written in symbolic form:
		\begin{center}
			($\forall x \in \mathbb{Z}$)($x$ is even).
		\end{center}
		\begin{itemize}
			\item[a.] Write this statement in English.\\ \pause
			\begin{center}
				For every integer, that integer is even.
			\end{center}
			\pause
			\item[b.] Is the statement true or false?  \pause False!  Counterexample: $x=1$. \pause
			\item[c.] How would you write the negation of this statement as an English sentence? \pause
			\begin{center}
				There exists an integer that is odd (not even).
			\end{center}
			\pause
			\item[d.] If possible, write your negation of this statement from part (c) symbolically
			(using a quantifier).\pause
			\[
			(\exists x \in \mathbb{Z})(x \text{ is odd}).
			\]
		\end{itemize}
	\end{enumerate}
\end{frame}

\begin{frame}{Preview Activity 2: Quantifiers}
	\begin{enumerate}
		\item[2.] Consider the following statement written in symbolic form:
		\begin{center}
			($\exists x \in \mathbb{Z}$)($x^3$ is greater than zero).
		\end{center}
		\begin{itemize}
			\item[a.] Write this statement in English.\\ \pause
			\begin{center}
				There exists an integer $x$ such that $x^3$ is greater than zero.
			\end{center}
			\pause
			\item[b.] Is the statement true or false?  \pause True!  For instance: $x=1 \to x^3 = 1 > 0$. \pause
			\item[c.] How would you write the negation of this statement as an English sentence? \pause
			\begin{center}
				For every integer x, $x^3 \leq 0$.
			\end{center}
			\pause
			\item[d.] If possible, write your negation of this statement from part (c) symbolically
			(using a quantifier).\pause
			\[
			(\forall x \in \mathbb{Z})(x^3 \leq 0).
			\]
		\end{itemize}
	\end{enumerate}
\end{frame}

%\begin{frame}{Preview Activity 2: Statements with Two Quantifiers}
%	\[
%	(\exists x \in \mathbb{R})(x \cdot y = 100).
%	\]
%	\begin{enumerate}
%		\item Explain why this sentence is not a statement\\[.1 in] \pause
%		\item If 5 is substituted for $y$, is the resulting sentence a statement?  If it is a statement, is it true or %false?\\[.1 in] \pause
%		\item If $-3$ is substituted for $y$, is the resulting sentence a statement?  If it is a statement, is it true or %false?\\[.1 in] \pause
%		\item If $\pi$ is substituted for $y$, is the resulting sentence a statement?  If it is a statement, is it true or %false?\\[.1 in] \pause
%		\item What is the truth set of the sentence? \pause $\{x \in \mathbb{R} | x \neq 0\}$
%	\end{enumerate}
%\end{frame}

%\begin{frame}{Preview Activity 2: Statements with Two Quantifiers}
%	\[
%	(\forall y \in \mathbb{R})\large((\exists x \in \mathbb{R})(x \cdot y = 100)\large).
%	\]
%	\begin{enumerate}
%		\item[6.a] Is this statement true or false? Explain.\\[.1 in]\pause
%		\item[6.b] Write this statement in the form of an English sentence. \pause
%		\begin{center}
%			For any real number $y$, there exists a real number $x$ such that $x \cdot y = 100$.
%		\end{center}
%		\pause
%		\[
%		(\exists x \in \mathbb{R})\large((\forall y \in \mathbb{R})(x \cdot y = 100)\large).
%		\]
%		
%		\item[7.a] Is this statement true or false? Explain.\\[.1 in]\pause
%		\item[7.b] Write this statement in the form of an English sentence.
%		\begin{center}
%			There exists a real number $x$ such that for any real number $y$, $x \cdot y = 100$.
%		\end{center}
%	\end{enumerate}
%\end{frame}

\begin{frame}{Progress Check 2.21: Negating Quantified Statements}
	For each of the following statements
	\begin{itemize}
		\item Write the statement in the form of an English sentence that does not use the symbols for quantifiers.
		\item Write the negation of the statement in a symbolic form that does not use the negation symbol.
		\item Write the negation of the statement in the form of an English sentence that does not use the symbols for quantifiers.
	\end{itemize}
\end{frame}

\begin{frame}{Progress Check 2.21.1: Negating Quantified Statements}
	\[(\forall a \in \mathbb{R})(a + 0 = a)\]
	\pause
	\begin{itemize}
		\item For any real number $a$, $a + 0 = a$.\pause
		\item \((\exists a \in \mathbb{R})(a + 0 \neq a)\) \pause
		\item There exists a real number $a$ such that $a + 0 \neq a$.
	\end{itemize}
\end{frame}

\begin{frame}{Progress Check 2.21.2: Negating Quantified Statements}
	\[\forall x \in \mathbb{R}, \sin(2x) = 2(\sin x)(\cos x)\]
	\pause
	\begin{itemize}
		\item For any real number $x$, $\sin(2x) = 2(\sin x)(\cos x)$.\pause
		\item \(\exists x \in \mathbb{R}, \sin(2x) \neq 2(\sin x)(\cos x)\) \pause
		\item There exists a real number $x$ such that $\sin(2x) \neq 2(\sin x)(\cos x)$.
	\end{itemize}
\end{frame}

\begin{frame}{Progress Check 2.21.3: Negating Quantified Statements}
	\[\forall x \in \mathbb{R}, \tan^2 x + 1 = \sec^2 x\]
	\pause
	\begin{itemize}
		\item For any real number $x$, $\tan^2 x + 1 = \sec^2 x$.\pause
		\item \(\exists x \in \mathbb{R}, \tan^2 x + 1 \neq \sec^2 x\) \pause
		\item There exists a real number $x$ such that $\tan^2 x + 1 \neq \sec^2 x$.
	\end{itemize}
\end{frame}

\begin{frame}{Progress Check 2.21.4: Negating Quantified Statements}
	\[\exists x \in \mathbb{Q}, x^2 - 3x - 7 = 0\]
	\pause
	\begin{itemize}
		\item There exists a rational number $x$ such that $x^2 - 3x - 7 = 0$.\pause
		\item \(\forall x \in \mathbb{Q}, x^2 - 3x - 7 \neq 0\) \pause
		\item For any rational number $x$, $x^2 - 3x - 7 \neq 0$.
	\end{itemize}
\end{frame}

\begin{frame}{Progress Check 2.21.5: Negating Quantified Statements}
	\[\exists x \in \mathbb{R}, x^2 + 1 = 0\]
	\pause
	\begin{itemize}
		\item There exists a real number $x$ such that $x^2 +1 = 0$.\pause
		\item \(\forall x \in \mathbb{R}, x^2 +1 \neq 0\) \pause
		\item For any real number $x$, $x^2 +1 \neq 0$.
	\end{itemize}
\end{frame}

\begin{frame}{Progress Check 2.22: Using Counterexamples}
	\begin{enumerate}
		\item For each integer $n$, $(n^2 + n + 1)$ is a prime number.\\[.1 in] \pause
		Counterexample: if $n = 4$, then $n^2 + n + 1 = 21 = 7 \cdot 3$, and thus, is not prime. \pause
		\item For each real number $x$, $2x^2 > x$.\\[.1 in]\pause
		Counterexample: If $x = 0.1$, then $2x^2 = 0.02 \not > 0.1 = x$.
	\end{enumerate}
\end{frame}

\begin{frame}{Progress Check 2.23}
	\begin{definition}
		An integer $n$ is a multiple of 3 provided that there exists an integer
		$k$ such that $n = 3k$.
	\end{definition}
	\pause
	\begin{enumerate}
		\item Write this definition in symbolic form using quantifiers by completing the following:
		\begin{center}
			An integer n is a multiple of 3 provided that ... \pause $\exists k \in \mathbb{Z}, n = 3k$.
		\end{center}
		\pause
		\item Give several examples of integers (including negative integers) that are multiples of 3.\pause
		\[-6,-3,0,3,6,9\]
		\pause
		\item Give several examples of integers (including negative integers) that are not multiples of 3.\pause
		\[-5,-4,-2,-1,1,2,4,5\]
	\end{enumerate}
\end{frame}

\begin{frame}{Progress Check 2.23}
	\begin{enumerate}
		\item[4.] Use the symbolic form of the definition of a multiple of 3 to complete the following sentence:
		\begin{center}
			An integer n is not a multiple of 3 provided that ... \pause $\forall k \in \mathbb{Z}, n \neq 3k$
		\end{center}
		\item[5.] Without using the symbols for quantifiers, complete the following sentence: 
		\begin{center}
			An integer n is not a multiple of 3 provided that ... \pause for any integer k, $n \neq 3k$.
		\end{center}
	\end{enumerate}
\end{frame}

\begin{frame}{Progress Check 2.24: Negating a Statement with Two Quantifiers}
	Write the negation of the statement
	\[
	(\forall x \in \mathbb{Z})(\forall y \in \mathbb{Z})(x+y = 0)
	\]
	in symbolic form and as a sentence written in English. \pause
	\[
	(\exists x \in \mathbb{Z})(\exists y \in \mathbb{Z})(x+y \neq 0)
	\]
	\pause
	There exists an integer x and there exists an integer y such that $x + y = 0$.
\end{frame} 

\end{document}


