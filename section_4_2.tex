\documentclass{beamer}
\usetheme{CambridgeUS}

\usepackage[utf8]{inputenc}
\usepackage[english]{babel}
\usepackage{amsthm}
\usepackage{tikzsymbols}
\usepackage{parskip}
\usepackage{multicol}

%These for for continuing numbering between frames
\newcounter{saveenumi}
\newcommand{\seti}{\setcounter{saveenumi}{\value{enumi}}}
\newcommand{\conti}{\setcounter{enumi}{\value{saveenumi}}}

%\newtheorem{theorem}{Theorem}[section]
%\newtheorem{corollary}{Corollary}[theorem]
%\newtheorem{lemma}[theorem]{Lemma}
%\theoremstyle{definition}
%\newtheorem{definition}{Definition}[section]
\newtheorem{prop}[theorem]{Proposition}

%Information to be included in the title page:
\title{Section 4.2}
\subtitle{Introduction to Proofs}
\author{Dr. Ryan Johnson}
\institute{Grace College}
\date{Spring 2021}


%%%%%%%%%%%%%%%%%%%%%%%%%%%%%%%%%%%%%%%%%%%%%%%%%%%%%%%%%%%%%%%%%%%
\begin{document}
\begin{frame}[plain]
    \maketitle
\end{frame}

\begin{frame}{The Jesus Creed}
\Large{
"Hear, O Israel, the Lord our God, the Lord is one.\\
You shall love the Lord your God\\
\;\; with all your heart, with all your soul,\\
\;\; with all your mind, and with all your strength.\\
The second is this: Love your neighbor as yourself.\\
There is no commandment greater than these."
}
\end{frame}

\begin{frame}{Preview Activity 1}
    \begin{definition}
    Definition. If n is a natural number, we define n factorial, denoted by $n!$ , to be the product of the first n natural numbers. In addition, we define $0!$ to be equal to 1.
    \end{definition}
    
    \begin{enumerate}
        \item Compute the values of $2^n$ and $n!$ for each natural number n with $1 \leq n \leq 7$. \pause
        \uplevel{Now let $P(n)$ be the open sentence, ``$n! > 2^n$.''} \pause
        \item Which of the statements $P(1)$ through $P(7)$ are true? \pause
        \item Based on the evidence so far, does the following proposition appear to be true or false? For each natural number n with $n \geq 4$ ,$n! > 2^n$.
        \item Explain why $(k+1) \cdot k! = (k+1)!$.
        \item Explain $k \geq 4 \implies (k+1) > 2 \implies (k+1)2^k > 2^{k+1}$
    \end{enumerate}
\end{frame}

\begin{frame}{Preview Activity 2}
    \begin{enumerate}
        \item Give examples of four natural numbers that are prime and four natural numbers
that are composite. \pause
        \item Write each of the natural numbers 20, 40, 50, and 150 as a product of prime
numbers. \pause
        \item Do you think that any composite number can be written as a product of prime
numbers? \pause
        \item Write a useful description of what it means to say that a natural number is a
composite number (other than saying that it is not prime). \pause
        \item Based on your work in Part (2), do you think it would be possible to use
induction to prove that any composite number can be written as a product of
prime numbers?
    \end{enumerate}
\end{frame}

\begin{frame}{Progress Check 4.8}
    Formulate a conjecture (with an appropriate quantifier) that can be used as an answer
to each of the following questions.
\begin{enumerate}
    \item When is $3^n$ greater than $1 + 2^n$?
    \item When is $2^n$ greater than $(n+1)^2$?
    \item What is $(1 + \frac{1}{n})^n$ greater than 2.5?
\end{enumerate}
\end{frame}

\end{document}


