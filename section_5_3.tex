\documentclass{beamer}
\usetheme{CambridgeUS}

\usepackage[utf8]{inputenc}
\usepackage[english]{babel}
\usepackage{amsthm}
\usepackage{tikzsymbols}
\usepackage{parskip}
\usepackage{multicol}

%These for for continuing numbering between frames
\newcounter{saveenumi}
\newcommand{\seti}{\setcounter{saveenumi}{\value{enumi}}}
\newcommand{\conti}{\setcounter{enumi}{\value{saveenumi}}}

%\newtheorem{theorem}{Theorem}[section]
%\newtheorem{corollary}{Corollary}[theorem]
%\newtheorem{lemma}[theorem]{Lemma}
%\theoremstyle{definition}
%\newtheorem{definition}{Definition}[section]
\newtheorem{prop}[theorem]{Proposition}

%Information to be included in the title page:
\title{Section 5.3}
\subtitle{Introduction to Proofs}
\author{Dr. Ryan Johnson}
\institute{Grace College}
\date{Spring 2021}


%%%%%%%%%%%%%%%%%%%%%%%%%%%%%%%%%%%%%%%%%%%%%%%%%%%%%%%%%%%%%%%%%%%
\begin{document}
\begin{frame}[plain]
    \maketitle
\end{frame}

\begin{frame}{The Jesus Creed}
\Large{
"Hear, O Israel, the Lord our God, the Lord is one.\\
You shall love the Lord your God\\
\;\; with all your heart, with all your soul,\\
\;\; with all your mind, and with all your strength.\\
The second is this: Love your neighbor as yourself.\\
There is no commandment greater than these."
}
\end{frame}

\begin{frame}[t]{Preview Activity 1}
    \begin{enumerate}
        \item Draw two general Venn diagrams for the sets A and B. On one, shade the
region that represents $(A \cup B)^c$, and on the other, shade the region that represents
$A^c \cap B^c$.
        \item Based on the Venn diagrams in Part (1), what appears to be the relationship?
        \seti
    \end{enumerate}
\end{frame}

\begin{frame}[t]{Preview Activity 1}
    \begin{enumerate}
        \conti
        \item Use one of DeMorgan’s Laws (Theorem 2.8 on page 48) to explain carefully
what it means to say that an element x is not in $A \cup B$. \pause
        \item What does it mean to say that an element x is in $A^c$? What does it mean to
say that an element x is in $B^c$? \pause
        \item Explain carefully what it means to say that an element x is in $A^c \cap B^c$ \pause
        \item Are \# 3 and \#5 similar?
        \item How are the sets related?
    \end{enumerate}
\end{frame}

\begin{frame}[t]{Preview Activity 2}
    \begin{enumerate}
        \item Let X, Y, and Z be statements.  Complete a truth table for
        \[\Big((X \to Y) \wedge (Y \to Z) \Big) \to (X \to Z)\]
        \seti
    \end{enumerate}
\end{frame}

\begin{frame}[t]{Preview Activity 2}
    \begin{enumerate}
        \conti
        \item Assume P, Q, and R are statements and that we have proven that the following conditional statements are true.
        \begin{itemize}
            \begin{multicols}{3}
            \item $P \to Q$
            \item $Q \to R$
            \item $R \to P$
            \end{multicols}
        \end{itemize}
        Explain why each of the following statments is true
        \begin{enumerate}
            \begin{multicols}{3}
                \item $P \leftrightarrow Q$
                \item $Q \leftrightarrow R$
                \item $R \leftrightarrow P$
            \end{multicols} 
        \end{enumerate}
    \end{enumerate}
\end{frame}

\begin{frame}[t]{Progress Check 5.19}
    Note: Theorem 5.18 on page 246 is important.
\begin{enumerate}
    \item Draw A Venn Diagram for the sets A, B, and C inside a universal set U.  Shade in $A \cup (B \cap C)$ and $(A \cup B) \cap (A \cup C)$.
    \item Is there a relationship?  Prove it.
\end{enumerate}
\end{frame}

\begin{frame}[t]{Class Activity}
    If time allows, prove $(A \cap B)^c = A^c \cup B^c$
\end{frame}

\end{document}


