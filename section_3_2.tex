\documentclass{beamer}
\usetheme{CambridgeUS}

\usepackage[utf8]{inputenc}
\usepackage[english]{babel}
\usepackage{amsthm}
\usepackage{tikzsymbols}
\usepackage{parskip}

%\newtheorem{theorem}{Theorem}[section]
%\newtheorem{corollary}{Corollary}[theorem]
%\newtheorem{lemma}[theorem]{Lemma}
%\theoremstyle{definition}
%\newtheorem{definition}{Definition}[section]
\newtheorem{prop}[theorem]{Proposition}

%Information to be included in the title page:
\title{Section 3,2}
\subtitle{Introduction to Proofs}
\author{Dr. Ryan Johnson}
\institute{Grace College}
\date{Spring 2021}

%%%%%%%%%%%%%%%%%%%%%%%%%%%%%%%%%%%%%%%%%%%%%%%%%%%%%%%%%%%%%%%%%%%
\begin{document}
\begin{frame}[plain]
    \maketitle
\end{frame}

\begin{frame}{The Jesus Creed}
\Large{
"Hear, O Israel, the Lord our God, the Lord is one.\\
You shall love the Lord your God\\
\;\; with all your heart, with all your soul,\\
\;\; with all your mind, and with all your strength.\\
The second is this: Love your neighbor as yourself.\\
There is no commandment greater than these."
}
\end{frame}

\begin{frame}{Preview Activity 1}
    Original statement proven in Section 1.2 exercise 3c. \par
    If n is an odd integer, then $n^2$ is an odd integer.\par
    New proposition \par
    For each integer $n$, if $n^2$ is an odd integer, then $n$ is an odd integer. \par
    \begin{enumerate}
    \pause
        \item Do you think it's true or false? \pause \;\; Answers may vary. \pause
        \item Know-show table.  \pause \;\; Trick question.  This should have been impossible. :-) \pause
        \item Write the contrapositive \pause \par
        For each integer $n$, if $n$ is even, then $n^2$ is even.
    \end{enumerate}
\end{frame}

\begin{frame}{Preview Activity 1}
    \begin{tabular}{| l | l | l |}
        \hline
        Step & Know & Reason \\ \hline
        P & $n$ is an even integer & Hypothesis \\ \hline
        & & \\ \hline
        & & \\ \hline
        & & \\ \hline
        & & \\ \hline
        & & \\ \hline
        Q & $n^2$ is an even integer & Definition of an even integer \\ \hline
    \end{tabular}
    \pause
    
    Since we proved the contrapostive, the original statement is also proved!
\end{frame}

\begin{frame}{Preview Activity 2}
    \begin{enumerate}
        \item Already did this one as a class activity
        \item So how do we prove a $P \leftrightarrow Q$ statement? \pause \; \; Break the proof into two parts.  First prove $P \to Q$, and then prove $Q \to P$.
        \item Have we proven \par
        For each integer n, n is an odd integer if and only if $n^2$ is an odd integer? \pause \par
        Yes!  Explanations may vary.
    \end{enumerate}
\end{frame}

\begin{frame}{Progress Check 3.8}
    Note: writing guidelines on page 105 \pause \par
    
    \begin{enumerate}
        \item In English, write the contrapositive of ``For all real numbers a and b, if $a \neq 0$ and $b \neq 0$, then $ab \neq 0$''.
        
        \pause
        Theorem 2.8 on page 48 says
        \[
        X \to (Y \vee Z) \equiv (X \wedge \neg Y) \to Z
        \]
        \pause
        \item Use the theorem above to write your answer to \#1 into an equivalent statement. \pause
        \item Prove the proposition.  Write your finished product on a whiteboard.
    \end{enumerate}
\end{frame}

\begin{frame}{In class}
    Section 3.2 \#4
\end{frame}

\end{document}


