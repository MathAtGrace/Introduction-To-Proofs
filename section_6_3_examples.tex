 \documentclass[12pt]{article}
\pagestyle{myheadings}
\usepackage{graphicx}

%Enter your name, the portfolio problem number, and the draft number.
\title{Johnson Section 6.3 Example 1}
\author{Ryan Johnson}

%Enter your name, the portfolio problem number, and the draft number.  This will be a heading on pages after the first page.
\markright{Ryan Johnson Problem 1 -- Draft 1}


\usepackage{amsmath,amssymb,amsthm,amsfonts,graphics}

%The following commands allow us to typeset theorems, propositions, definitions, etc.
\theoremstyle{plain}
\newtheorem{theorem}{Theorem}
\newtheorem{lemma}[theorem]{Lemma}
\newtheorem{corollary}[theorem]{Corollary}
\newtheorem{proposition}[theorem]{Proposition}
\newtheorem*{definition}{Definition}

\renewcommand{\qedsymbol}{\ensuremath{\blacksquare}}
\newcommand{\N}{\mathbb{N}}
\newcommand{\Z}{\mathbb{Z}}
\newcommand{\Q}{\mathbb{Q}}
\newcommand{\R}{\mathbb{R}}


\begin{document}
\maketitle

%Enter your email address.
\begin{center}
\textbf{email address: johnsor@grace.edu}
\end{center}

\begin{proposition}
Let $h : \mathbb{R} \to \mathbb{R}$ be defined by $h = x^2 - 3x$.  Then $h$ is not injective.
\end{proposition}

\begin{proof}
Consider that
\[
h(0) = 0^2 - 3\cdot 0 = 0 = 3^2 - 3\cdot 3 = h(3).
\]
Then $h$ is not injective by definition.
\end{proof}

\begin{proposition}
Let $F : \mathbb{Z} \to \mathbb{Z}$ be defined by $F(m) = 3m+2$.  Then $F$ is injective.
\end{proposition}

\begin{proof}
We let $m, n \in \mathbb{Z}$ and assume that $F(m) = F(n)$ and will prove that $m=n$.  Since $F(m) = F(n)$, substitution gives us
\[
3m+2 = 3n +2.
\]
Subtracting 2 from both sides and then dividing both sides by 3 gives us
\[
m=n.
\]
Thus, $F$ is injective by definition.
\end{proof}

%%%%%%%%%%%%%%%%%%%%%%%%%%%%%%%%
\newpage

\begin{proposition}
Let $F : \mathbb{Z} \to \mathbb{Z}$ be defined by $F(m) = 3m + 2$.  Then $F$ is not surjective.
\end{proposition}

\begin{proof}
This will be a proof by contradiction.  Assume $F$ is surjective.  In other words, assume that for all integers $y$, there exists an integer $x$ such that $F(x) = y$.  Consider when $y = 3$.  Then $F(x) = 3$ implies
\[
3x + 2 = 3.
\]
Solving for $x$ gives us
\begin{align*}
    3x &= 1\\
    x &= \frac{1}{3}.
\end{align*}
However, we are assuming $x$ is an integer.  Absurd!  Thus, $F$ is not surjective.
\end{proof}

\end{document}


