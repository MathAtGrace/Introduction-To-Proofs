\documentclass{beamer}
\usetheme{CambridgeUS}

\usepackage[utf8]{inputenc}
\usepackage[english]{babel}
\usepackage{amsthm}
\usepackage{tikzsymbols}
\usepackage{parskip}
\usepackage{multicol}

%\newtheorem{theorem}{Theorem}[section]
%\newtheorem{corollary}{Corollary}[theorem]
%\newtheorem{lemma}[theorem]{Lemma}
%\theoremstyle{definition}
%\newtheorem{definition}{Definition}[section]
\newtheorem{prop}[theorem]{Proposition}

%Information to be included in the title page:
\title{Section 2.2}
\subtitle{Introduction to Proofs}
\author{Dr. Ryan Johnson}
\institute{Grace College}
\date{Spring 2021}

%%%%%%%%%%%%%%%%%%%%%%%%%%%%%%%%%%%%%%%%%%%%%%%%%%%%%%%%%%%%%%%%%%%
\begin{document}
\begin{frame}[plain]
    \maketitle
\end{frame}

\begin{frame}{The Jesus Creed}
\Large{
"Hear, O Israel, the Lord our God, the Lord is one.\\
You shall love the Lord your God\\
\;\; with all your heart, with all your soul,\\
\;\; with all your mind, and with all your strength.\\
The second is this: Love your neighbor as yourself.\\
There is no commandment greater than these."
}
\end{frame}

\begin{frame}{Preview Activity 1: Logically Equivalent Statements}
	\begin{definition}
		Two expressions are \textbf{logically equivalent} provided that they have
		the same truth value for all possible combinations of truth values for all variables
		appearing in the two expressions. In this case, we write $X \equiv Y$ and say
		that X and Y are logically equivalent.
	\end{definition}
	1. Complete the truth tables for $\neg (P \wedge Q)$ and $\neg P \vee \neg Q$. \pause
	
	\begin{center}
		\begin{tabular}{lr}
			\begin{tabular}{|l|l|l|l|}
				\hline
				P & Q & $P \wedge Q$ & $\neg (P \wedge Q)$\\ \hline
				T&T&& \\ \hline
				T&F&& \\ \hline
				F&T&& \\ \hline
				F&T&& \\ \hline
			\end{tabular}
			&
			\begin{tabular}{|l|l|l|l|l|}
				\hline
				P & Q & $\neg P$ & $\neg Q$ & $\neg P \vee \neg Q$\\ \hline
				T&T&&& \\ \hline
				T&F&&& \\ \hline
				F&T&&& \\ \hline
				F&F&&& \\ \hline
			\end{tabular}
		\end{tabular}
	\end{center}
	\pause
	2. Are the expressions $\neg (P \wedge Q)$ and $\neg P \vee \neg Q$ logically equivalent?
\end{frame} 

\begin{frame}{Preview Activity 1: Logically Equivalent Statements}
	\begin{definition}
		Two expressions are \textbf{logically equivalent} provided that they have
		the same truth value for all possible combinations of truth values for all variables
		appearing in the two expressions. In this case, we write $X \equiv Y$ and say
		that X and Y are logically equivalent.
	\end{definition}
	3. Suppose that the statement ``I will play golf and I will mow the lawn'' is false.
	Then its negation is true. Write the negation of this statement in the form of
	a disjunction. Does this make sense?\\[.1 in] \pause
	
	``I will not play golf or I will not mow the lawn.''
\end{frame}

%\begin{frame}{Preview Activity 1: Conditional Statements}
%	Imagine a parent making the following two statements.\\[.1 in]
%	\begin{tabular}{lp{3 in}}
%		Statement 1: & If you do not clean your room, then you cannot watch TV.\\
%		Statement 2: & You clean your room or you cannot watch TV.
%	\end{tabular}
%	\begin{enumerate}
%		\item[4] Do these two statements mean the same thing? Explain.\\[2 in]
%	\end{enumerate}
%\end{frame}

\begin{frame}{Preview Activity 1: Conditional Statements}
	Imagine a parent making the following two statements.
	
	\begin{tabular}{lp{3 in}}
		Statement 1: & If you do not clean your room, then you cannot watch TV.\\
		Statement 2: & You clean your room or you cannot watch TV.
	\end{tabular}
	\begin{enumerate}
		\item[4.] Let $P$ be ``you do not clean your room,” and let Q be “you cannot watch TV.''
		Use these to translate Statement 1 and Statement 2 into symbolic forms.\\[.1 in] \pause
		\item[5.] Construct a truth table for each of the expressions you determined in 2.
		Are the expressions logically equivalent? \pause
	\end{enumerate}
	\begin{center}
		\begin{tabular}{lr}
			\begin{tabular}{|l|l|l|}
				\hline
				P & Q & $P \to Q$\\ \hline
				T&T& \\ \hline
				T&F& \\ \hline
				F&T& \\ \hline
				F&T& \\ \hline
			\end{tabular}
			&
			\begin{tabular}{|l|l|l|l|}
				\hline
				P & Q & $\neg P$ & $\neg P \vee Q$\\ \hline
				T&T&& \\ \hline
				T&F&& \\ \hline
				F&T&& \\ \hline
				F&F&& \\ \hline
			\end{tabular}
		\end{tabular}
	\end{center}
\end{frame}

\begin{frame}{Preview Activity 3: Conditional Statements}
	Imagine a parent making the following two statements.
	
	\begin{tabular}{lp{3 in}}
		Statement 1: & If you do not clean your room, then you cannot watch TV.\\
		Statement 2: & You clean your room or you cannot watch TV.
	\end{tabular}
	\begin{enumerate}
		\item[6.] Assume that Statement 1 and Statement 2 are false. In this case, what is
		the truth value of P and what is the truth value of Q? Now, write a true
		statement in symbolic form that is a conjunction and involves P and Q.
	\end{enumerate}
\end{frame}

\begin{frame}{Preview Activity 3: Conditional Statements}
	Imagine a parent making the following two statements.
	
	\begin{tabular}{lp{3 in}}
		Statement 1: & If you do not clean your room, then you cannot watch TV.\\
		Statement 2: & You clean your room or you cannot watch TV.
	\end{tabular}
	\begin{enumerate}
		\item[7.] Write a truth table for the (conjunction) statement in Part 4. and compare it
		to a truth table for $\neg(P \to Q)$.  What do you observe? \pause
	\end{enumerate}
\begin{center}
	\begin{tabular}{lr}
		\begin{tabular}{|l|l|l|l|}
			\hline
			P & Q & $\neg Q$ & $P \wedge \neg Q$\\ \hline
			T&T&& \\ \hline
			T&F&& \\ \hline
			F&T&& \\ \hline
			F&T&& \\ \hline
		\end{tabular}
		&
		\begin{tabular}{|l|l|l|l|}
			\hline
			P & Q & $P \to Q$ & $\neg ( P \to Q)$\\ \hline
			T&T&& \\ \hline
			T&F&& \\ \hline
			F&T&& \\ \hline
			F&F&& \\ \hline
		\end{tabular}
	\end{tabular}
\end{center}
\end{frame}

\begin{frame}{Preview Activity 2: Converse and Contrapositive}
	\begin{definition}
		The \textbf{converse} of the conditional statement $P \to Q$ is the conditional
		statement $Q \to P$.\\
		
		The \textbf{contrapositive} of the conditional statement $P \to Q$ is the conditional
		statement $\neg Q \to \neg  P$.
	\end{definition}
	\begin{enumerate}
		\item For the following, the variable x represents a real number. Label each of the
		following statements as true or false.
		\begin{multicols}{2}
			\begin{enumerate}
				\item[a.] If $x = 3$, then $x^2 = 9$.
				\item[b.] If $x^2 = 9$, then $x = 3$.
				\item[c.] If $x^2 \not = 9$, then $x \not = 3$.
				\item[d.] If $x \not = 3$, then $x^2 \not = 9$. \pause
			\end{enumerate}
		\end{multicols}
		\item Which statement in the list of conditional statements in (1) is the converse
		of Statement 1a? Which is the contrapositive?
	\end{enumerate}
\end{frame}



\begin{frame}{Notes}
	\begin{itemize}
		\item Theorem 2.5 (De Morgan's Laws) is really important.\\[.2 in] \pause
		\item Logical Equivalences of Conditional Statements.\\[.2 in] \pause
		\item p46 - words in bold: ``The negation of a conditional statement is not another conditional statement.''\\[.2 in] \pause
		\item Methods of Establishing Logical Equivalences\\[.2 in] \pause
		\item Check Moodle for Logical Equivalencies PDF.
	\end{itemize}
\end{frame}

\begin{frame}{Progress Check 2.8: Working with Logical Equivalency}
	\begin{enumerate}
		\item Although it is possible to use truth tables to show that $P \to (Q \vee R)$ is
		logically equivalent to $(P \vee \neg Q) \to R$, we instead use previously proven
		logical equivalencies to prove this logical equivalency. In this case, it may
		be easier to start working with $(P \wedge \neg Q) \to R$.  Start with
		\[
		(P \wedge \neg Q) \to R \equiv \neg (P \wedge \neg Q) \vee R,
		\]
		which is justified by the logical equivalency established in Preview Activity 3-3. Continue by using one of De Morgan’s Laws on $\neg (P \wedge \neg Q)$.
	\end{enumerate}
	\vspace{2 in}
\end{frame}

\begin{frame}{Progress Check 2.8: Working with Logical Equivalency}
	\begin{enumerate}
		\item[2.] Let a and b be integers. Suppose we are trying to prove the following:
		\begin{itemize}
			\item If 3 is a factor of $a \cdot b$, then 3 is a factor of a or 3 is a factor of b.
		\end{itemize}
		Explain why we will have proven this statement if we prove the following:
		\begin{itemize}
			\item If 3 is a factor of $a \cdot b$ and 3 is not a factor of a, then 3 is a factor of b.
		\end{itemize}
	\end{enumerate}
	\vspace{2 in}
\end{frame}

\begin{frame}{Activity 2.10: Working with Logical Equivalency}
	Suppose we are trying to prove the
	following for integers x and y:
	\begin{center}
		If $x \cdot y$ is even, then x is even or y is even.
	\end{center}
	We notice that we can write this statement in the following symbolic form:
	\[
	P \to (Q \vee R),
	\]
	where P is ``$x \cdot y$ is even.'' Q is ``x is even.'' and R is ``y is even.''
	\begin{enumerate}
		\item Manipulate the symbolic form using contrapositive and De Morgan's Laws.\\[.5 in]\pause
		\item Explain why the answer to part 1 is
		\begin{center}
			If x is odd and y is odd, then $x \cdot y$ is odd. \pause
		\end{center}
		Hey!  We already proved this!  It's Theorem 1.6 in Section 1.2.
	\end{enumerate}
\end{frame}

\end{document}


