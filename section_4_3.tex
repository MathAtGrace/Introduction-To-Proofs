\documentclass{beamer}
\usetheme{CambridgeUS}

\usepackage[utf8]{inputenc}
\usepackage[english]{babel}
\usepackage{amsthm}
\usepackage{tikzsymbols}
\usepackage{parskip}
\usepackage{multicol}

%These for for continuing numbering between frames
\newcounter{saveenumi}
\newcommand{\seti}{\setcounter{saveenumi}{\value{enumi}}}
\newcommand{\conti}{\setcounter{enumi}{\value{saveenumi}}}

%\newtheorem{theorem}{Theorem}[section]
%\newtheorem{corollary}{Corollary}[theorem]
%\newtheorem{lemma}[theorem]{Lemma}
%\theoremstyle{definition}
%\newtheorem{definition}{Definition}[section]
\newtheorem{prop}[theorem]{Proposition}

%Information to be included in the title page:
\title{Section 4.3}
\subtitle{Introduction to Proofs}
\author{Dr. Ryan Johnson}
\institute{Grace College}
\date{Spring 2021}


%%%%%%%%%%%%%%%%%%%%%%%%%%%%%%%%%%%%%%%%%%%%%%%%%%%%%%%%%%%%%%%%%%%
\begin{document}
\begin{frame}[plain]
    \maketitle
\end{frame}

\begin{frame}{The Jesus Creed}
\Large{
"Hear, O Israel, the Lord our God, the Lord is one.\\
You shall love the Lord your God\\
\;\; with all your heart, with all your soul,\\
\;\; with all your mind, and with all your strength.\\
The second is this: Love your neighbor as yourself.\\
There is no commandment greater than these."
}
\end{frame}

\begin{frame}{Preview Activity 1}
    \[b_1 = 16 \ \ \ \text{ and } \ \ \ b_{n+1} = \frac{1}{2} b_n \]
    \begin{enumerate}
        \item Calculate $b_4$ through $b_{10}$. What seems to be happening to the values of $b_n$ as n gets larger? \pause
        \item Define a sequence recursively as follows:
        \[
        T_1 = 16, \text{ and for each } n \in \mathbb{N}, T_{n+1} = 16 + \frac{1}{2}T_n.
        \]
        Calculate $T_3$ through $T_{10}$. \pause
        \begin{align*}
            a_1 &= a, \text{ and for each } n \in \mathbb{N}, a_{n+1} = r \cdot a_n
            S_1 &= a, \text{ and for each } n \in \mathbb{N}, S_{n+1} = r \cdot S_n
        \end{align*}
        \item Determine formulas (in terms of a and r) for $a_2$ through $a_6$. What do you think an is equal to (in terms of a, r, and n)? \pause
        \item Determine formulas (in terms of a and r) for $S_2$ through $S_6$. What do you think an is equal to (in terms of a, r, and n)? 
        \seti
    \end{enumerate}
\end{frame}

\begin{frame}{Preview Activity 1}
    \begin{enumerate}
        \conti
        \[a_1 = 1\ \ \ \text{ and } a_{n+1} = (n+1)a_n\]
        \item Calculate $a_3$ through $a_6$ \pause
        \item $a_{20}$? $a_{100}$? \pause
        \item Can we calculate $a_n$ for any natural number?
    \end{enumerate}
\end{frame}

\begin{frame}{Preview Activity 2}
    $f_1 = 1$, $f_2 = 1$ and for each $n \in \mathbb{N}$ $f_{n+2} = f_{n+1} + f_n$.
    \begin{enumerate}
        \item Calculate $f_6$ through $f_{20}$ \pause
        \item Which Fibonacci numbers are even?  Which are multiples of 3?
        \item partial sums of Fibonacci numbers?
        \item Record any other observations about the values of the Fibonacci numbers
or any patterns that you observe in the sequence of Fibonacci numbers. If
necessary, compute more Fibonacci numbers.
    \end{enumerate}
\end{frame}

\begin{frame}{Progress Check 4.12}
For each natural number n, the Fibonacci number $f_{3n}$ is an
even natural number.    
\end{frame}

\end{document}


