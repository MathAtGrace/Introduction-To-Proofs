\documentclass{beamer}
\usetheme{CambridgeUS}

\usepackage[utf8]{inputenc}
\usepackage[english]{babel}
\usepackage{amsthm}
\usepackage{tikzsymbols}
\usepackage{parskip}
\usepackage{multicol}

%These for for continuing numbering between frames
\newcounter{saveenumi}
\newcommand{\seti}{\setcounter{saveenumi}{\value{enumi}}}
\newcommand{\conti}{\setcounter{enumi}{\value{saveenumi}}}

%\newtheorem{theorem}{Theorem}[section]
%\newtheorem{corollary}{Corollary}[theorem]
%\newtheorem{lemma}[theorem]{Lemma}
%\theoremstyle{definition}
%\newtheorem{definition}{Definition}[section]
\newtheorem{prop}[theorem]{Proposition}

%Information to be included in the title page:
\title{Section 6.1}
\subtitle{Introduction to Proofs}
\author{Dr. Ryan Johnson}
\institute{Grace College}
\date{Spring 2021}


%%%%%%%%%%%%%%%%%%%%%%%%%%%%%%%%%%%%%%%%%%%%%%%%%%%%%%%%%%%%%%%%%%%
\begin{document}
\begin{frame}[plain]
    \maketitle
\end{frame}

\begin{frame}{The Jesus Creed}
\Large{
"Hear, O Israel, the Lord our God, the Lord is one.\\
You shall love the Lord your God\\
\;\; with all your heart, with all your soul,\\
\;\; with all your mind, and with all your strength.\\
The second is this: Love your neighbor as yourself.\\
There is no commandment greater than these."
}
\end{frame}

\begin{frame}[t]{Preview Activity 1}
Which of the following equations can be used to define a function with $x \in \mathbb{R}$
as the input and $y \in \mathbb{R}$ as the output?
    \begin{enumerate}
    \begin{multicols}{2}
        \item Yes
        \item No (two outputs)
        \item Yes
        \item Yes
        \item No (zero or two outputs)
        \item Yes
        \item No (doesn't work for $x=1$)
    \end{multicols}
    \end{enumerate}
\end{frame}

\begin{frame}{Preview Activity 2}
    \begin{enumerate}
        \item b the birthday function
        \begin{enumerate}
            \item Because every person has exactly one birthday.
            \item $b(\text{Andrew Wiles}) = \text{April 11, 1953}$
            \item True
            \item False
        \end{enumerate}
        \seti
    \end{enumerate}
\end{frame}

\begin{frame}{Preview Activity 2}
    \begin{enumerate}
        \conti
        \item Calculate $s(k)$ for each natural number k from 1 through 15
        \item Does there exist a natural number n such that $s(k) = 5$? Justify your conclusion.
        \item Is it possible to find two different natural numbers m and n such that $s(m) = s(n)$? Explain.
        \item Use response to answer questions
        \begin{enumerate}
            \item For each $m \in \mathbb{N}$, there exists a natural number n such that $s(n) = m$.
            \item For all $m,n \in \mathbb{N}$, if $m \neq n$, then $s(m) \neq s(n)$.
        \end{enumerate}
    \end{enumerate}
\end{frame}

\begin{frame}{Progress Check 6.1}
    \begin{definition}
    Definition. A function from a set A to a set B is a rule that associates with
each element x of the set A exactly one element of the set B. A function from
A to B is also called a mapping from A to B.
    \end{definition}
    
     Let $f : \mathbb{R} \to \mathbb{R}$ be defined by $f(x) = x^2 - 5x$ and $g : \mathbb{Z} \to \mathbb{Z}$ be defined by $g(m) = m^2 - 5m$.
     \begin{enumerate}
         \item Determine $f(-3)$ and $f(\sqrt{8}$.
         \item Determine $g(2)$ and $g(-2)$
         \item Determine the set of all preimages of 6 for the function f.
        \item Determine the set of all preimages of 6 for the function g.
        \item Determine the set of all preimages of 2 for the function f.
        \item Determine the set of all preimages of 2 for the function g.
     \end{enumerate}    
\end{frame}

\begin{frame}{Progress Check 6.2}
    Let b be the function that assigns to each person his or her birthday (month
and day).
\begin{enumerate}
    \item What is the domain of this function?
    \item What is a codomain for this function?
    \item What is the range of this function?
\end{enumerate}
Let s be the function that associates with each natural number the sum of its
distinct natural number factors.
\begin{enumerate}
    \item What is the domain of this function?
    \item What is a codomain for this function?
    \item Is the range equal to the codomain?
\end{enumerate}
\end{frame}



\end{document}


