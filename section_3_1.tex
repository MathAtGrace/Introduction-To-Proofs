\documentclass{beamer}
\usetheme{CambridgeUS}

\usepackage[utf8]{inputenc}
\usepackage[english]{babel}
\usepackage{amsthm}
\usepackage{tikzsymbols}
\usepackage{parskip}

%\newtheorem{theorem}{Theorem}[section]
%\newtheorem{corollary}{Corollary}[theorem]
%\newtheorem{lemma}[theorem]{Lemma}
%\theoremstyle{definition}
%\newtheorem{definition}{Definition}[section]
\newtheorem{prop}[theorem]{Proposition}

%Information to be included in the title page:
\title{Section 3.1}
\subtitle{Introduction to Proofs}
\author{Dr. Ryan Johnson}
\institute{Grace College}
\date{Spring 2021}

%%%%%%%%%%%%%%%%%%%%%%%%%%%%%%%%%%%%%%%%%%%%%%%%%%%%%%%%%%%%%%%%%%%
\begin{document}
\begin{frame}[plain]
    \maketitle
\end{frame}

\begin{frame}{The Jesus Creed}
\Large{
"Hear, O Israel, the Lord our God, the Lord is one.\\
You shall love the Lord your God\\
\;\; with all your heart, with all your soul,\\
\;\; with all your mind, and with all your strength.\\
The second is this: Love your neighbor as yourself.\\
There is no commandment greater than these."
}
\end{frame}

\begin{frame}{Preview Activity 1: Definitions of Divides, Divisor, Multiple}
\begin{definition}
	A nonzero integer $m$ \textbf{divides} an integer $n$ provided that there is
	an integer q such that $n = m \cdot q$. We also say that m is a \textbf{divisor} of n, m is
	a \textbf{factor} of n, and n is a \textbf{multiple} of m. The integer 0 is not a divisor of any
	integer. If a and b are integers and $a \neq 0$, we frequently use the notation $a \; | \; b$
	as a shorthand for ``a divides b.''
\end{definition}
\begin{enumerate}
	\item Give three different examples of three integers where the first integer divides the second integer and the second integer divides the third integer\pause
	\[3, 6, 30 \hspace{.5 in} 6 = 2 \cdot 3,\; 30 = 5 \cdot 6 \] \pause \vspace{-.1 in}
	\item In your examples in Part (1), is there any relationship between the first and the third integer?  Write your conjecture in the form of a conditional statement with appropriate quantifiers. \pause
	\[
		(\forall a \in \mathbb{Z})(\forall b \in \mathbb{Z})(\forall c \in \mathbb{Z})((a\;|\;b) \wedge (b\;|\;c) \to (a \;|\; c))
	\]
\end{enumerate}
\end{frame}

\begin{frame}{Preview Activity 1: Definitions of Divides, Divisor, Multiple}
	\begin{enumerate}
		\setcounter{enumi}{2}
		\item Give several examples of two integers where the first integer does not divide the second integer. \pause
		\[
		2,3 \hspace{.5 in} 4,6 \hspace{.5 in} 12,16
		\]
		\pause \vspace{-.3 in}
		\item According to the definition of ``divides,'' does the integer 10 divide the integer 0?  That is, is 10 a divisor of 0?  Explain. \pause
		\begin{center}
			Yes, because $10 \cdot 0 = 0$. \pause
		\end{center}
		\item Use the definition of ``divides'' to complete the following in symbolic form: ``The nonzero integer m does not divide the integer n means that...'' \pause
		\[
			\forall k \in \mathbb{Z}, n \neq m \cdot k
		\]
		\pause \vspace{-.3 in}
		\item Use the definition of ``divides'' to complete the following without using symbols for quantifiers: ``The nonzero integer m does not divide the integer n means that...'' \pause
		\begin{center}
			For all integers $k$, $n \neq m \cdot k$.
		\end{center}
	\end{enumerate}
\end{frame}

\begin{frame}{Preview Activity 2: A Proposition about Multiples of 3}
	\begin{enumerate}
		\item Multiply several pairs of integer where at least one of the integers is a multiple of 3.\pause
		\[2 \cdot 3 = 6 \hspace{.5 in} 6 \cdot 5 = 30 \hspace{.5 in} 9 \cdot 8 = 72\]
		\pause \vspace{-.3 in}
		\item Do your examples in Part (1) support the following proposition or prove the proposition false?  Explain.
		\begin{prop}
			For all integers $m$ and $n$, if $m$ is a multiple of 3, then $m \cdot n$ is a multiple of 3. \pause
		\end{prop}
		\item Write the conditional statement in the proposition in symbolic form. \pause
		\[
		(\forall m \in \mathbb{Z})(\forall n \in \mathbb{Z})(\;(3 \; | \; m) \to (3 \; | \; mn))
		\]
	\end{enumerate}
\end{frame}

\begin{frame}{Preview Activity 2: A Proposition about Multiples of 3}
	\begin{enumerate}
		\setcounter{enumi}{3}
		\item What is the hypothesis of the conditional statement?  What is the conclusion?\\[.1 in] \pause
		Hypothesis: ``m is a multiple of 3''\\Conclusion: ``mn is a multiple of 3.''
		\item What is the definition of a multiple of 3? \pause
		\begin{center}
			$x \in \mathbb{Z}$ is a multiple of 3 if there exists a $y \in mathbb{Z}$ such that $x = 3y$.
		\end{center}
		\pause
		\item Construct a know-show table for this proposition.
	\end{enumerate}
\end{frame}

\begin{frame}{Preview Activity 2: A Proposition about Multiples of 3}
	\begin{tabular}{|l|p{2.5 in}|p{1 in}|}
		\hline
		P & Let $m, n \in \mathbb{Z}$ such that m is a multiple of 3. & Hypothesis\\ \hline
		& & \\ \hline
		& & \\ \hline
		& & \\ \hline
		Q & mn is a multiple of 3 & \\ \hline
	\end{tabular}
\end{frame}

\begin{frame}{Preview Activity 3, part 1}
	Suppose that it is currently Tuesday.
	\begin{enumerate}
		\item What day will it be 3 days from now?\pause
		\item What day will it be 10 days from now?\pause
		\item What day will it be 17 days from now? What day will it be 24 days
		from now? \pause
		\item Find several other natural numbers x such that it will be Friday x days
		from now.
		\item Create a list (in increasing order) of the numbers 3; 10; 17; 24, and the
		numbers you generated in Part (4). Pick any two numbers from this
		list and subtract one from the other. Repeat this several times. \pause
		\item What do the numbers you obtained in Part (5) have in common?
	\end{enumerate}
\end{frame}

\begin{frame}{Preview Activity 3, part 2}
	Suppose that we are using a twelve-hour clock with no distinction between
	a.m. and p.m. Also, suppose that the current time is 5:00.
	\begin{enumerate}
		\item What time will it be 4 hours from now?\pause
		\item What time will it be 16 hours from now? What time will it be 28 hours
		from now? \pause
		\item Find several other natural numbers x such that it will be 9:00 x hours
		from now.\pause
		\item Create a list (in increasing order) of the numbers 4; 16; 28, and the
		numbers you generated in Part (3). Pick any two numbers from this
		list and subtract one from the other. Repeat this several times. \pause
		\item What do the numbers you obtained in Part (4) have in common?
	\end{enumerate}
\end{frame}

\begin{frame}{Preview Activity 3, part 3}
	This is a continuation of Part 1. Suppose that it is currently Tuesday.
	\begin{enumerate}
		\item What day was it 4 days ago?\pause
		\item What day was it 11 days ago? What day was it 18 days ago? \pause
		\item Find several other natural numbers x such that it was Friday x days
		ago. \pause
		\item Create a list (in increasing order) consisting of the numbers
		$-18, -11, -4$, the opposites of the numbers you generated in Part (3c)
		and the positive numbers in the list from Part (1e). Pick any two numbers
		from this list and subtract one from the other. Repeat this several
		times. \pause
		\item What do the numbers you obtained in Part (4) have in common?
	\end{enumerate}
\end{frame}

\begin{frame}{Progress Check 3.2: A Property of Divisors}
	\begin{enumerate}
		\item Give at least four different examples of integers a, b, and c with $a \neq 0$ such
		that a divides b and a divides c.\\[.1 in]\pause
		\item For each example in Part (1), calculate the sum $b+c$. Does the integer a
		divide the sum $b + c$?\\[.1 in]\pause
		\item Formulate a conjecture converning the relationship between the integer a and the sum $b+c$.\\[.1 in]\pause
		\item Construct a know-show table for a proof of the conjecture in part (3).
	\end{enumerate}
\end{frame}

\begin{frame}{Progress Check 3.2: A Property of Divisors}
	\begin{tabular}{|l|p{2.5 in}|p{1.5 in}|}
		\hline
		P & Let a,b, and c be integers such that a divides b and a divides c & Hypothesis\\ \hline
		&&\\\hline
		&&\\\hline
		&&\\\hline
		&&\\\hline
		Q & a divides $b+c$. & \\ \hline
	\end{tabular}
\end{frame}

\end{document}


