\documentclass{beamer}
\usetheme{CambridgeUS}

\usepackage[utf8]{inputenc}
\usepackage[english]{babel}
\usepackage{amsthm}
\usepackage{tikzsymbols}
\usepackage{parskip}
\usepackage{multicol}

%These for for continuing numbering between frames
\newcounter{saveenumi}
\newcommand{\seti}{\setcounter{saveenumi}{\value{enumi}}}
\newcommand{\conti}{\setcounter{enumi}{\value{saveenumi}}}

%\newtheorem{theorem}{Theorem}[section]
%\newtheorem{corollary}{Corollary}[theorem]
%\newtheorem{lemma}[theorem]{Lemma}
%\theoremstyle{definition}
%\newtheorem{definition}{Definition}[section]
\newtheorem{prop}[theorem]{Proposition}

%Information to be included in the title page:
\title{Section 4.1}
\subtitle{Introduction to Proofs}
\author{Dr. Ryan Johnson}
\institute{Grace College}
\date{Spring 2021}


%%%%%%%%%%%%%%%%%%%%%%%%%%%%%%%%%%%%%%%%%%%%%%%%%%%%%%%%%%%%%%%%%%%
\begin{document}
\begin{frame}[plain]
    \maketitle
\end{frame}

\begin{frame}{The Jesus Creed}
\Large{
"Hear, O Israel, the Lord our God, the Lord is one.\\
You shall love the Lord your God\\
\;\; with all your heart, with all your soul,\\
\;\; with all your mind, and with all your strength.\\
The second is this: Love your neighbor as yourself.\\
There is no commandment greater than these."
}
\end{frame}

\begin{frame}{Preview Activity 1}
    For each natural number n, let P(n) be the following open sentence:
    \[4 \text{ divides }\ (5^h-1). \] \pause
\begin{enumerate}
    \item Does this open sentence become a true statement when n = 1? That is, is 1 in the truth set of P(n)? \pause
    \item Does this open sentence become a true statement when n = 2? That is, is 2 in the truth set of P(n)? \pause
    \item Choose at least four more natural numbers and determine whether the open sentence is true or false for each of your choices.
    \seti
\end{enumerate}
\end{frame}

\begin{frame}{Preview Activity 1}
    For each natural number n, let $Q(x)$ be the following open sentence
    \[
    1^2 + 2^2 + 3^2 + ... + n^2 = \frac{n(n+1)(2n+1)}{6}.
    \] \pause
    \begin{enumerate}
        \conti
        \item Does Q(n) become a true statement when $n = 1,2,3$? \pause
        \item Choose at least four more numbers.
    \end{enumerate}
\end{frame}

\begin{frame}{Preview Activity 2}
    \begin{definition}
    A set T that is a subset of $\mathbb{Z}$ is an inductive set provided that for each integer k, if $k \in T$ , then $k+1 \in T$ .
    \end{definition}
    
    \begin{enumerate}
        \item Carefully explain what it means to say that a subset T of the integers $\mathbb{Z}$ is not an inductive set. This description should use an existential quantifier.
        \begin{enumerate}
        \begin{multicols}{2}
            \item $A = \{1,2,...20\}$ \pause
            \item $\mathbb{N}$ \pause
            \item $B = \{n \in \mathbb{N} \ | \ n \geq 5\}$ \pause
            \item $S = \{n \in \mathbb{N} \ | \ n \geq -3\}$ \pause
            \item $R = \{n \in \mathbb{N} \ | \ n \leq 100\}$ \pause
            \item $\mathbb{Z}$
            \item The odd natural numbers
            \end{multicols}
        \end{enumerate}
        \item Assume that $T \subseteq \mathbb{N}$ and assume that $1 \in \mathbb{N}$ and that T is an inductive set.  Are 2,3,4, 100 in T?  Is $T = \mathbb{N}$?
    \end{enumerate}
\end{frame}

\begin{frame}{Progress Check 4.1}
    Suppose that T is an inductive subset of the integers. Which of the following
statements are true, which are false, and for which ones is it not possible to tell?
\begin{enumerate}
    \begin{multicols}{2}
        \item $1 \in T$ and $5 \in T$
        \item If $1 \in T$ then $5 \in T$
        \item $5 \not \in T$ then $2 \not \in T$
        \item $\forall k \in T$, if $k \in T$, then $k+7 \in T$.
        \item For each integer k, $k \not \in T$ or $k+1 \in T$
        \item There exists an integer k such that $k \in T$ and $k+1 \not \in T$
        \item For each integer k, if $k+1 \in T$, then $k \in T$.
        \item For each integer k, if $k+1 \not \in T$, then $k \not \in T$
    \end{multicols}
\end{enumerate}
\end{frame}

\begin{frame}{Frame Title}
    Note: Procedure for a Proof by Mathematical Induction on page 174\\
    
    In class, prove 
    \[
    1^2 + 2^2 + 3^2 + ... + n^2 = \frac{n(n+1)(2n+1}{6}.
    \]
    
    Then, student prove
    \[
    1 + 2 + 3 + ... + n = \frac{n(n+1)}{2}.
    \]
    
    If time allows, also have students prove preview activity 2
    
    HW \#6
\end{frame}


\end{document}


