\documentclass{beamer}
\usetheme{CambridgeUS}

\usepackage[utf8]{inputenc}
\usepackage[english]{babel}
\usepackage{amsthm}
\usepackage{tikzsymbols}
\usepackage{parskip}
\usepackage{multicol}

%These for for continuing numbering between frames
\newcounter{saveenumi}
\newcommand{\seti}{\setcounter{saveenumi}{\value{enumi}}}
\newcommand{\conti}{\setcounter{enumi}{\value{saveenumi}}}

%\newtheorem{theorem}{Theorem}[section]
%\newtheorem{corollary}{Corollary}[theorem]
%\newtheorem{lemma}[theorem]{Lemma}
%\theoremstyle{definition}
%\newtheorem{definition}{Definition}[section]
\newtheorem{prop}[theorem]{Proposition}

%Information to be included in the title page:
\title{Section 5.4}
\subtitle{Introduction to Proofs}
\author{Dr. Ryan Johnson}
\institute{Grace College}
\date{Spring 2021}


%%%%%%%%%%%%%%%%%%%%%%%%%%%%%%%%%%%%%%%%%%%%%%%%%%%%%%%%%%%%%%%%%%%
\begin{document}
\begin{frame}[plain]
    \maketitle
\end{frame}

\begin{frame}{The Jesus Creed}
\Large{
"Hear, O Israel, the Lord our God, the Lord is one.\\
You shall love the Lord your God\\
\;\; with all your heart, with all your soul,\\
\;\; with all your mind, and with all your strength.\\
The second is this: Love your neighbor as yourself.\\
There is no commandment greater than these."
}
\end{frame}

\begin{frame}[t]{Preview Activity 1}
    \begin{enumerate}
        \item List six different elements of the truth set (often called the solution set) of
the open sentence with two variables $2x+3y=12$.
        \item Sketch the graph.  What does it show?
        \item Write a description of the solution set S of the equation  $2x+3y=12$ using
set builder notation.
    \end{enumerate}
\end{frame}

\begin{frame}{Preview Activity 2}
    Let $A = \{1,2,3,\}$ and $B = \{a,b\}$.
    \begin{enumerate}
        \item Is $(3,a) \in A \times B$?
        \item Is $(3,a) \in A \times A$?
        \item Is $(3,1) \in A \times A$?
        \item List all elements of $A \times B$
        \item List all elements of $A \times A$
        \item What does it mean for $(x,y)$ to not be in $C \times D$?
    \end{enumerate}
\end{frame}

\begin{frame}{Progress Check 5.23}
    Let $A = \{1,2,3\}$, $T = \{1,2\}$, $B = \{a,b\}$, and $C = \{a,c\}$. Use the roster method to list the elments of each set and describe any relationships you see.
    \begin{multicols}{2}
    \begin{enumerate}
        \item $A \times B$
        \item $T \times B$
        \item $A \times C$
        \item $A \times (B \cap C)$
        \item $(A \times B) \cap (A \times C)$
        \item $A \times (B \cup C)$
        \item $(A \times B) \cup (A \times C)$
        \item $A \times (B - C)$
        \item $(A \times B) - (A \times C)$
        \item $B \times A$
    \end{enumerate}
    \end{multicols}
\end{frame}

\begin{frame}{Progress Check 5.24}
    For
    \[
    A = [0,2] \;\;\; T = (1,2) \;\;\; B = [2,4) \;\;\; C = (3,5]
    \]
    Draw a graph of each of the following subsets of the Cartesian plane and
write each subset using set builder notation.  List any relationships.
\begin{multicols}{2}
    \begin{enumerate}
        \item $A \times B$
        \item $T \times B$
        \item $A \times C$
        \item $A \times (B \cap C)$
        \item $(A \times B) \cap (A \times C)$
        \item $A \times (B \cup C)$
        \item $(A \times B) \cup (A \times C)$
        \item $A \times (B - C)$
        \item $(A \times B) - (A \times C)$
        \item $B \times A$
    \end{enumerate}
    \end{multicols}
\end{frame}

\begin{frame}[t]{Class Activity}
    If time allows, prove Theorem 5.25, Part (7): If $T \subseteq A$, then $T \times B \subseteq A \times B$
\end{frame}

\end{document}


