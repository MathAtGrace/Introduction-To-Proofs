\documentclass{beamer}
\usetheme{CambridgeUS}

\usepackage[utf8]{inputenc}
\usepackage[english]{babel}
\usepackage{amsthm}
\usepackage{tikzsymbols}
\usepackage{parskip}

%\newtheorem{theorem}{Theorem}[section]
%\newtheorem{corollary}{Corollary}[theorem]
%\newtheorem{lemma}[theorem]{Lemma}
%\theoremstyle{definition}
%\newtheorem{definition}{Definition}[section]
\newtheorem{prop}[theorem]{Proposition}

%Information to be included in the title page:
\title{Section 2.1}
\subtitle{Introduction to Proofs}
\author{Dr. Ryan Johnson}
\institute{Grace College}
\date{Spring 2021}

%%%%%%%%%%%%%%%%%%%%%%%%%%%%%%%%%%%%%%%%%%%%%%%%%%%%%%%%%%%%%%%%%%%
\begin{document}
\begin{frame}[plain]
    \maketitle
\end{frame}

\begin{frame}{The Jesus Creed}
\Large{
"Hear, O Israel, the Lord our God, the Lord is one.\\
You shall love the Lord your God\\
\;\; with all your heart, with all your soul,\\
\;\; with all your mind, and with all your strength.\\
The second is this: Love your neighbor as yourself.\\
There is no commandment greater than these."
}
\end{frame}

\begin{frame}{Preview Activity 1}
	We will use the following two statements for all of this Preview Activity:
	\begin{itemize}
		\item P is the statement ``It is raining.''
		\item Q is the statement ``Daisy is playing golf.''
	\end{itemize}
	\begin{enumerate}
		\item When P is true (it is raining) and Q is true (Daisy is playing golf).\\[.1 in]
		\begin{tabular}{ll}
			(a) $(P \wedge Q)$ & It is raining and Daisy is playing golf.\\[.1 in]
			(b) $(P \vee Q)$ & It is raining or Daisy is playing golf.\\[.1 in]
			(c) $(P \to Q)$ & If it is raining, then Daisy is playing golf.\\[.1 in]
			(d) $(\neg P)$ & It is not raining.
		\end{tabular}
	\end{enumerate}
\end{frame}

\begin{frame}{Preview Activity 1}
	We will use the following two statements for all of this Preview Activity:
	\begin{itemize}
		\item P is the statement ``It is raining.''
		\item Q is the statement ``Daisy is playing golf.''
	\end{itemize}
	\begin{enumerate}
		\item[2.]  When P is true (it is raining) and Q is false (Daisy is not playing golf).\\[.1 in]
		\begin{tabular}{ll}
			(a) $(P \wedge Q)$ & It is raining and Daisy is playing golf.\\[.1 in]
			(b) $(P \vee Q)$ & It is raining or Daisy is playing golf.\\[.1 in]
			(c) $(P \to Q)$ & If it is raining, then Daisy is playing golf.\\[.1 in]
			(d) $(\neg P)$ & It is not raining.
		\end{tabular}
	\end{enumerate}
\end{frame}

\begin{frame}{Preview Activity 1}
	We will use the following two statements for all of this Preview Activity:
	\begin{itemize}
		\item P is the statement ``It is raining.''
		\item Q is the statement ``Daisy is playing golf.''
	\end{itemize}
	\begin{enumerate}
		\item[3.]  When P is false (it is not raining) and Q is true (Daisy is playing golf).\\[.1 in]
		\begin{tabular}{ll}
			(a) $(P \wedge Q)$ & It is raining and Daisy is playing golf.\\[.1 in]
			(b) $(P \vee Q)$ & It is raining or Daisy is playing golf.\\[.1 in]
			(c) $(P \to Q)$ & If it is raining, then Daisy is playing golf.\\[.1 in]
			(d) $(\neg P)$ & It is not raining.
		\end{tabular}
	\end{enumerate}
\end{frame}

\begin{frame}{Preview Activity 1}
	We will use the following two statements for all of this Preview Activity:
	\begin{itemize}
		\item P is the statement ``It is raining.''
		\item Q is the statement ``Daisy is playing golf.''
	\end{itemize}
	\begin{enumerate}
		\item[4.]  When P is false (it is not raining) and Q is false (Daisy is not playing golf).\\[.1 in]
		\begin{tabular}{ll}
			(a) $(P \wedge Q)$ & It is raining and Daisy is playing golf.\\[.1 in]
			(b) $(P \vee Q)$ & It is raining or Daisy is playing golf.\\[.1 in]
			(c) $(P \to Q)$ & If it is raining, then Daisy is playing golf.\\[.1 in]
			(d) $(\neg P)$ & It is not raining.
		\end{tabular}
	\end{enumerate}
\end{frame}

\begin{frame}{Preview Activity 2 (Truth Tables)}
	\begin{center} 
	\begin{tabular}{|c|c|}
		\hline
		P & not P\\ \hline
		T & \\ \hline
		F & \\ \hline
	\end{tabular}

	\vspace{.5 in}
	
	\begin{tabular}{|c|c|c|}
		\hline
		P & Q & P and Q\\ \hline
		T & T & \\ \hline
		T & F & \\ \hline
		F & T & \\ \hline
		F & F & \\ \hline
	\end{tabular}
	\end{center}
\end{frame}

\begin{frame}{Preview Activity 2 (Truth Tables)}
	\begin{center} 
		\begin{tabular}{|c|c|c|}
			\hline
			P & Q & P or Q\\ \hline
			T & T & \\ \hline
			T & F & \\ \hline
			F & T & \\ \hline
			F & F & \\ \hline
		\end{tabular}
		
		\vspace{.5 in}
		
		\begin{tabular}{|c|c|c|}
			\hline
			P & Q & $P \to Q$\\ \hline
			T & T & \\ \hline
			T & F & \\ \hline
			F & T & \\ \hline
			F & F & \\ \hline
		\end{tabular}
	\end{center}
\end{frame}

\begin{frame}{Preview Activity 2 (Truth Tables)}
	\begin{center} 
		\begin{tabular}{|c|c|c|c|c|}
			\hline
			P & Q & not P & not Q & not $Q \to $ not P\\ \hline
			T & T &&& \\ \hline
			T & F &&& \\ \hline
			F & T &&& \\ \hline
			F & F &&& \\ \hline
		\end{tabular}
	\end{center}
\end{frame}

\begin{frame}{Things to note in section 2.1}
	\begin{itemize}
		\item Definitions of logical operator and compound statement.\\[.1 in] \pause
		\item Table 2.1 on page 31\\[.1 in] \pause
		\item Math ``or'' is ``inclusive or.''\\[.1 in] \pause
		\item Table 2.2 on page 32\\[.1 in] \pause
		\item Bottom of page 32: different ways we say $P \to Q$ in English.
	\end{itemize}
\end{frame}

\begin{frame}{Progress Check 2.1}
	Recall that a quadrilateral is a four-sided polygon. Let S represent the following
	true conditional statement:
	\begin{center}
		If a quadrilateral is a square, then it is a rectangle.
	\end{center}
	Write this conditional statement in English using
	\begin{enumerate}
		\item the word ``whenever.''\\[.1 in] \pause
		\begin{center}
			Whenever a quadrilateral is a square, it is a rectangle.\\[.1 in] \pause
		\end{center}
		\item the phrase ``only if.''\\[.1 in] \pause
		\begin{center}
			A quadrilateral is a square, only if it is a rectangle.\\[.1 in] 
		\end{center}
	\end{enumerate}
\end{frame}

\begin{frame}{Progress Check 2.1}
	Recall that a quadrilateral is a four-sided polygon. Let S represent the following
	true conditional statement:
	\begin{center}
		If a quadrilateral is a square, then it is a rectangle.
	\end{center}
	Write this conditional statement in English using
	\begin{enumerate}
		\item[3.] the phrase ``is necessary for.''\\[.1 in] \pause
		\begin{center}
			A quadrilateral being a rectangle is necessary for it being a square.\\[.1 in] \pause
		\end{center}
		\item[4.] the phrase ``is sufficient for.''\\[.1 in] \pause
		\begin{center}
			A quadrilateral being a square is sufficient for it being a rectangle.\\[.1 in]
		\end{center}
	\end{enumerate}
\end{frame}

\begin{frame}{Progress Check 2.2 (Constructing Truth Tables)}
	1. $P \wedge \neg Q$
\end{frame}

\begin{frame}{Progress Check 2.2 (Constructing Truth Tables)}
	2. $\neg(P \wedge Q)$
\end{frame}

\begin{frame}{Progress Check 2.2 (Constructing Truth Tables)}
	3. $\neg P \wedge \neg Q$
\end{frame}

\begin{frame}{Progress Check 2.2 (Constructing Truth Tables)}
	4. $\neg P \vee \neg Q$
\end{frame}

\begin{frame}{Progress Check 2.3 (The Truth Table for the Biconditional Statement)}
	\begin{center}
		Complete a truth table for $\large((Q \to P) \wedge (P \to Q)\large)$.
	\end{center}
	\vspace{3 in}
\end{frame} 

\begin{frame}{Acitivity 2.4 (Working with Conditional Statements)}
	\begin{tabular}{|l|c|c|c|}
		\hline
		English Form & Hypothesis & Conclusion & Symbolic Form\\[.05 in] \hline
		If P, then Q &&&\\[.05 in] \hline
		Q only if P  &&&\\[.05 in] \hline
		P is necessary for Q  &&&\\[.05 in] \hline
		P is sufficient for Q  &&&\\[.05 in] \hline
		P implies Q  &&&\\[.05 in] \hline
		P only if Q  &&&\\[.05 in] \hline
		P if Q  &&&\\[.05 in] \hline
		If Q then P  &&&\\[.05 in] \hline
		If $\neg Q$ then $\neg p$. &&&\\[.05 in] \hline
		If P, then $Q \wedge R$.  &&&\\[.05 in] \hline
		If $P \vee Q$, then R.  &&&\\[.05 in] \hline
	\end{tabular}
\end{frame}

\end{document}


